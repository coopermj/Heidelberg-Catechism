%% grace-6.tex - Part 2f: Of Man's Redemption (Baptism)
%% Questions 69-77, Lord's Days XXVI-XXVIII

%% ============================================================
%% LORD'S DAY XXVI
%% ============================================================

\lordsday{26}

\catechismqa{69}{%
How does holy baptism signify and seal to you that the one sacrifice of Christ on the cross benefits you?%
}{%
In this way: Christ instituted this outward washing and with it gave the promise that, as surely as water washes away the dirt from the body, so certainly his blood and Spirit wash away the impurity of my soul, that is, all my sins.%
}

\catechismqa{70}{%
What does it mean to be washed with Christ's blood and Spirit?%
}{%
To be washed with Christ's blood means to receive forgiveness of sins from God, through grace, because of Christ's blood, poured out for us in his sacrifice on the cross. To be washed with his Spirit means to be renewed by the Holy Spirit and sanctified to be members of Christ, so that more and more we become dead to sin and lead a holy and blameless life.%
}

\catechismqa{71}{%
Where has Christ promised that we are washed with his blood and Spirit as surely as we are washed with the water of baptism?%
}{%
In the institution of baptism, where he says: Go therefore and make disciples of all nations, baptizing them in the name of the Father and of the Son and of the Holy Spirit. He who believes and is baptized will be saved, but he who does not believe will be condemned. This promise is repeated where Scripture calls baptism the washing of regeneration and the washing away of sins.%
}

%% ============================================================
%% LORD'S DAY XXVII
%% ============================================================

\vspace{0.5em}
\lordsday{27}

\catechismqa{72}{%
Does this outward washing with water itself wash away sins?%
}{%
No, only the blood of Jesus Christ and the Holy Spirit cleanse us from all sins.%
}

\catechismqa{73}{%
Why then does the Holy Spirit call baptism the washing of regeneration and the washing away of sins?%
}{%
God speaks in this way for a good reason: He wants to teach us that the blood and Spirit of Christ remove our sins just as water takes away dirt from the body. But, more important, he wants to assure us by this divine pledge and sign that we are as truly cleansed from our sins spiritually as we are bodily washed with water.%
}

\catechismqa{74}{%
Should infants also be baptized?%
}{%
Yes. Infants as well as adults belong to God's covenant and congregation. Through Christ's blood the redemption from sin and the Holy Spirit, who works faith, are promised to them no less than to adults. Therefore, by baptism, as sign of the covenant, they must be grafted into the Christian church and distinguished from the children of unbelievers. This was done in the old covenant by circumcision, in place of which baptism was instituted in the new covenant.%
}

%% ============================================================
%% LORD'S DAY XXVIII
%% ============================================================

\vspace{0.5em}
\lordsday{28}

\catechismqa{75}{%
How does the Lord's Supper signify and seal to you that you share in Christ's one sacrifice on the cross and all his benefits?%
}{%
In this way: Christ has commanded me and all believers to eat of this broken bread and drink of this cup in remembrance of him, and has joined with this command these promises: First, as surely as I see with my eyes the bread of the Lord broken for me and the cup given to me, so surely was his body offered for me and his blood poured out for me on the cross. Second, as surely as I receive from the hand of the minister and taste with my mouth the bread and the cup of the Lord as sure signs of Christ's body and blood, so surely does he himself nourish and refresh my soul to everlasting life with his crucified body and poured-out blood.%
}

\catechismqa{76}{%
What does it mean to eat the crucified body of Christ and to drink his poured-out blood?%
}{%
It means not only to embrace with a believing heart all the sufferings and the death of Christ, and thereby to receive forgiveness of sins and eternal life, but also to be so united more and more to his sacred body by the Holy Spirit, who dwells both in Christ and in us, that, although Christ is in heaven and we are on earth, we are nevertheless flesh of his flesh and bone of his bones, and we live and are governed forever by one Spirit, as the members of our body are by one soul.%
}

\catechismqa{77}{%
Where has Christ promised that he will nourish and refresh believers with his body and blood as surely as they eat of this broken bread and drink of this cup?%
}{%
In the institution of the Lord's Supper: The Lord Jesus on the night when he was betrayed took bread, and when he had given thanks, he broke it, and said, This is my body which is for you. Do this in remembrance of me. In the same way also the cup, after supper, saying, This cup is the new covenant in my blood. Do this, as often as you drink it, in remembrance of me. For as often as you eat this bread and drink the cup, you proclaim the Lord's death until he comes. This promise is repeated by Paul where he says: The cup of blessing which we bless, is it not a participation in the blood of Christ? The bread which we break, is it not a participation in the body of Christ? Because there is one bread, we who are many are one body, for we all partake of the one bread.%
}
