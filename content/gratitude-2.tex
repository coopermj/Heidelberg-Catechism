%% gratitude-2.tex - Part 3b: Of Thankfulness (The Law Continued)
%% Questions 96-107, Lord's Days XXXV-XL

%% ============================================================
%% LORD'S DAY XXXV
%% ============================================================

\lordsday{35}

\catechismqa{96}{%
What does God require in the second commandment?%
}{%
We are not to make an image of God in any way, nor to worship him in any other manner than he has commanded in his Word.%
}

\catechismqa{97}{%
May we then not make any image at all?%
}{%
God cannot and may not be visibly portrayed in any way. Creatures may be portrayed, but God forbids us to make or have any images of them in order to worship them or to serve God through them.%
}

\catechismqa{98}{%
But may images not be tolerated in the churches as books for the laity?%
}{%
No, for we should not be wiser than God. He wants his people to be taught not by means of dumb images but by the living preaching of his Word.%
}

%% ============================================================
%% LORD'S DAY XXXVI
%% ============================================================

\vspace{0.5em}
\lordsday{36}

\catechismqa{99}{%
What is required in the third commandment?%
}{%
We are not to blaspheme or to abuse the name of God by cursing, perjury, or unnecessary oaths, nor to share in such horrible sins by being silent bystanders. Rather, we must use the holy name of God only with fear and reverence, so that we may rightly confess him, call upon him, and praise him in all our words and works.%
}

\catechismqa{100}{%
Is the blaspheming of God's name by swearing and cursing such a grievous sin that God is angry also with those who do not prevent and forbid it as much as they can?%
}{%
Certainly, for no sin is greater or provokes God's wrath more than the blaspheming of his name. That is why he commanded it to be punished with death.%
}

%% ============================================================
%% LORD'S DAY XXXVII
%% ============================================================

\vspace{0.5em}
\lordsday{37}

\catechismqa{101}{%
But may we swear an oath by the name of God in a godly manner?%
}{%
Yes, when the civil authorities require it of their subjects, or when necessity demands it, in order to maintain and promote fidelity and truth, to the glory of God and the well-being of our neighbour. Such oath-taking is grounded in God's Word and was therefore rightly used by saints in the Old and the New Testament.%
}

\catechismqa{102}{%
May we also swear by saints or other creatures?%
}{%
No. A lawful oath is a calling upon God, who alone knows the heart, to bear witness to the truth, and to punish me if I swear falsely. No creature deserves such honour.%
}

%% ============================================================
%% LORD'S DAY XXXVIII
%% ============================================================

\vspace{0.5em}
\lordsday{38}

\catechismqa{103}{%
What does God require in the fourth commandment?%
}{%
First, that the ministry of the gospel and the schools be maintained and that, especially on the day of rest, I diligently attend the church of God to hear God's Word, to use the sacraments, to call publicly upon the Lord, and to give Christian offerings for the poor. Second, that all the days of my life I rest from my evil works, let the Lord work in me through his Holy Spirit, and so begin in this life the eternal Sabbath.%
}

%% ============================================================
%% LORD'S DAY XXXIX
%% ============================================================

\vspace{0.5em}
\lordsday{39}

\catechismqa{104}{%
What does God require in the fifth commandment?%
}{%
That I show all honour, love, and faithfulness to my father and mother and to all those in authority over me, submit myself with due obedience to their good instruction and discipline, and also bear patiently with their weaknesses and shortcomings, since it is God's will to govern us by their hand.%
}

%% ============================================================
%% LORD'S DAY XL
%% ============================================================

\vspace{0.5em}
\lordsday{40}

\catechismqa{105}{%
What does God require in the sixth commandment?%
}{%
I am not to dishonour, hate, injure, or kill my neighbour by thoughts, words, or gestures, and much less by deeds, whether personally or through another; rather, I am to put away all desire of revenge. Moreover, I am not to harm or recklessly endanger myself. Therefore, also, the authorities arm themselves to prevent murder.%
}

\catechismqa{106}{%
But does this commandment speak only of killing?%
}{%
By forbidding murder God teaches us that he hates the root of murder, such as envy, hatred, anger, and desire of revenge, and that he regards all these as murder.%
}

\catechismqa{107}{%
Is it enough, then, that we do not kill our neighbour in any such way?%
}{%
No. When God condemns envy, hatred, and anger, he requires us to love our neighbour as ourselves, to show patience, peace, gentleness, mercy, and friendliness toward him, to protect him from harm as much as we can, and also to do good even to our enemies.%
}
