%% grace-1.tex - Part 2a: Of Man's Redemption (God the Son)
%% Questions 12-52, Lord's Days V-XIX

%% ============================================================
%% LORD'S DAY V
%% ============================================================

\lordsday{5}

\catechismqa{12}{%
Since, according to God's righteous judgment we deserve temporal and eternal punishment, how can we escape this punishment and be again received into favour?%
}{%
God demands that his justice be satisfied. Therefore we must make full payment, either by ourselves or through another.%
}

\catechismqa{13}{%
Can we by ourselves make this payment?%
}{%
Certainly not. On the contrary, we daily increase our debt.%
}

\catechismqa{14}{%
Can any mere creature pay for us?%
}{%
No. In the first place, God will not punish another creature for the sin which man has committed. Furthermore, no mere creature can sustain the burden of God's eternal wrath against sin and deliver others from it.%
}

\catechismqa{15}{%
What kind of mediator and deliverer must we seek?%
}{%
One who is a true and righteous man, and yet more powerful than all creatures; that is, one who is at the same time true God.%
}

%% ============================================================
%% LORD'S DAY VI
%% ============================================================

\vspace{0.5em}
\lordsday{6}

\catechismqa{16}{%
Why must he be a true and righteous man?%
}{%
He must be a true man because the justice of God requires that the same human nature which has sinned should pay for sin. He must be a righteous man because one who himself is a sinner cannot pay for others.%
}

\catechismqa{17}{%
Why must he at the same time be true God?%
}{%
He must be true God so that by the power of his divine nature he might bear in his human nature the burden of God's wrath, and might obtain for us and restore to us righteousness and life.%
}

\catechismqa{18}{%
But who is that Mediator who at the same time is true God and a true and righteous man?%
}{%
Our Lord Jesus Christ, who has become for us wisdom from God---that is, our righteousness, holiness and redemption.%
}

\catechismqa{19}{%
From where do you know this?%
}{%
From the holy gospel, which God himself first revealed in Paradise. Later, he had it proclaimed by the patriarchs and prophets, and foreshadowed by the sacrifices and other ceremonies of the law. Finally, he had it fulfilled through his only Son.%
}

%% ============================================================
%% LORD'S DAY VII
%% ============================================================

\vspace{0.5em}
\lordsday{7}

\catechismqa{20}{%
Are all men, then, saved by Christ just as they perished through Adam?%
}{%
No. Only those are saved who by a true faith are grafted into Christ and accept all his benefits.%
}

\catechismqa{21}{%
What is true faith?%
}{%
True faith is a sure knowledge whereby I accept as true all that God has revealed to us in his Word. At the same time it is a firm confidence that not only to others, but also to me, God has granted forgiveness of sins, everlasting righteousness, and salvation, out of mere grace, only for the sake of Christ's merits. This faith the Holy Spirit works in my heart by the gospel.%
}

\catechismqa{22}{%
What, then, must a Christian believe?%
}{%
All that is promised us in the gospel, which the articles of our catholic and undoubted Christian faith teach us in a summary.%
}

\catechismqa{23}{%
What are these articles?%
}{%
I believe in God the Father almighty, Creator of heaven and earth. I believe in Jesus Christ, his only-begotten Son, our Lord; he was conceived by the Holy Spirit, born of the virgin Mary; suffered under Pontius Pilate, was crucified, dead, and buried; he descended into hell. On the third day he arose from the dead; he ascended into heaven, and sits at the right hand of God the Father almighty; from there he will come to judge the living and the dead. I believe in the Holy Spirit; I believe a holy catholic Christian church, the communion of saints; the forgiveness of sins; the resurrection of the body; and the life everlasting.%
}

%% ============================================================
%% LORD'S DAY VIII
%% ============================================================

\vspace{0.5em}
\lordsday{8}

\catechismqa{24}{%
How are these articles divided?%
}{%
Into three parts: the first is about God the Father and our creation; the second about God the Son and our redemption; the third about God the Holy Spirit and our sanctification.%
}

\catechismqa{25}{%
Since there is only one God, why do you speak of three persons, Father, Son, and Holy Spirit?%
}{%
Because God has so revealed himself in his Word that these three distinct persons are the one, true, eternal God.%
}

%% ============================================================
%% LORD'S DAY IX
%% ============================================================

\vspace{0.5em}
\lordsday{9}

\catechismqa{26}{%
What do you believe when you say: I believe in God the Father almighty, Creator of heaven and earth?%
}{%
That the eternal Father of our Lord Jesus Christ, who out of nothing created heaven and earth and all that is in them, and who still upholds and governs them by his eternal counsel and providence, is, for the sake of Christ his Son, my God and my Father. In him I trust so completely as to have no doubt that he will provide me with all things necessary for body and soul, and will also turn to my good whatever adversity he sends me in this life of sorrow. He is able to do so as almighty God, and willing also as a faithful Father.%
}

%% ============================================================
%% LORD'S DAY X
%% ============================================================

\vspace{0.5em}
\lordsday{10}

\catechismqa{27}{%
What do you understand by the providence of God?%
}{%
God's providence is his almighty and ever present power, whereby, as with his hand, he still upholds heaven and earth and all creatures, and so governs them that leaf and blade, rain and drought, fruitful and barren years, food and drink, health and sickness, riches and poverty, indeed, all things, come to us not by chance but by his fatherly hand.%
}

\catechismqa{28}{%
What does it benefit us to know that God has created all things and still upholds them by his providence?%
}{%
We can be patient in adversity, thankful in prosperity, and with a view to the future we can have a firm confidence in our faithful God and Father that no creature shall separate us from his love; for all creatures are so completely in his hand that without his will they cannot so much as move.%
}

%% ============================================================
%% LORD'S DAY XI
%% ============================================================

\vspace{0.5em}
\lordsday{11}

\catechismqa{29}{%
Why is the Son of God called Jesus, that is, Saviour?%
}{%
Because he saves us from all our sins, and because salvation is not to be sought or found in anyone else.%
}

\catechismqa{30}{%
Do those who seek their salvation or well-being in saints, in themselves, or anywhere else, also believe in the only Saviour Jesus?%
}{%
No. Though they boast of him in words, they in fact deny the only Saviour Jesus. For one of two things must be true: either Jesus is not a complete Saviour, or those who by true faith accept this Saviour must find in him all that is necessary for their salvation.%
}

%% ============================================================
%% LORD'S DAY XII
%% ============================================================

\vspace{0.5em}
\lordsday{12}

\catechismqa{31}{%
Why is he called Christ, that is, Anointed?%
}{%
Because he has been ordained by God the Father, and anointed with the Holy Spirit, to be our chief Prophet and Teacher, who has fully revealed to us the secret counsel and will of God concerning our redemption; our only High Priest, who by the one sacrifice of his body has redeemed us, and who continually intercedes for us before the Father; and our eternal King, who governs us by his Word and Spirit, and who defends and preserves us in the redemption obtained for us.%
}

\catechismqa{32}{%
Why are you called a Christian?%
}{%
Because I am a member of Christ by faith and thus share in his anointing, so that I may as prophet confess his name, as priest present myself a living sacrifice of thankfulness to him, and as king fight with a free and good conscience against sin and the devil in this life, and hereafter reign with him eternally over all creatures.%
}

%% ============================================================
%% LORD'S DAY XIII
%% ============================================================

\vspace{0.5em}
\lordsday{13}

\catechismqa{33}{%
Why is he called God's only-begotten Son, since we also are children of God?%
}{%
Because Christ alone is the eternal, natural Son of God. We, however, are children of God by adoption, through grace, for Christ's sake.%
}

\catechismqa{34}{%
Why do you call him our Lord?%
}{%
Because he has ransomed us, body and soul, from all our sins, not with silver or gold but with his precious blood, and has freed us from all the power of the devil to make us his own possession.%
}

%% ============================================================
%% LORD'S DAY XIV
%% ============================================================

\vspace{0.5em}
\lordsday{14}

\catechismqa{35}{%
What do you confess when you say: He was conceived by the Holy Spirit, born of the virgin Mary?%
}{%
The eternal Son of God, who is and remains true and eternal God, took upon himself true human nature from the flesh and blood of the virgin Mary, through the working of the Holy Spirit. Thus he is also the true seed of David, and like his brothers in every respect, yet without sin.%
}

\catechismqa{36}{%
What benefit do you receive from the holy conception and birth of Christ?%
}{%
He is our Mediator, and with his innocence and perfect holiness covers, in the sight of God, my sin, in which I was conceived and born.%
}

%% ============================================================
%% LORD'S DAY XV
%% ============================================================

\vspace{0.5em}
\lordsday{15}

\catechismqa{37}{%
What do you confess when you say that he suffered?%
}{%
During all the time he lived on earth, but especially at the end, Christ bore in body and soul the wrath of God against the sin of the whole human race. Thus, by his suffering, as the only atoning sacrifice, he has redeemed our body and soul from everlasting damnation, and obtained for us the grace of God, righteousness, and eternal life.%
}

\catechismqa{38}{%
Why did he suffer under Pontius Pilate as judge?%
}{%
Though innocent, Christ was condemned by an earthly judge, and so he freed us from the severe judgment of God that was to fall on us.%
}

\catechismqa{39}{%
Does it have a special meaning that Christ was crucified and did not die in a different way?%
}{%
Yes. Thereby I am assured that he took upon himself the curse which lay on me, for a crucified one was cursed by God.%
}

%% ============================================================
%% LORD'S DAY XVI
%% ============================================================

\vspace{0.5em}
\lordsday{16}

\catechismqa{40}{%
Why was it necessary for Christ to humble himself even unto death?%
}{%
Because of the justice and truth of God satisfaction for our sins could be made in no other way than by the death of the Son of God.%
}

\catechismqa{41}{%
Why was he buried?%
}{%
His burial testified that he had really died.%
}

\catechismqa{42}{%
Since Christ has died for us, why do we still have to die?%
}{%
Our death is not a payment for our sins, but it puts an end to sin and is an entrance into eternal life.%
}

\catechismqa{43}{%
What further benefit do we receive from Christ's sacrifice and death on the cross?%
}{%
Through Christ's death our old nature is crucified, put to death, and buried with him, so that the evil desires of the flesh may no longer reign in us, but that we may offer ourselves to him as a sacrifice of thankfulness.%
}

\catechismqa{44}{%
Why is there added: He descended into hell?%
}{%
In my greatest sorrows and temptations I may be assured and comforted that my Lord Jesus Christ, by his unspeakable anguish, pain, terror, and agony, which he endured throughout all his sufferings but especially on the cross, has delivered me from the anguish and torment of hell.%
}

%% ============================================================
%% LORD'S DAY XVII
%% ============================================================

\vspace{0.5em}
\lordsday{17}

\catechismqa{45}{%
How does Christ's resurrection benefit us?%
}{%
First, by his resurrection he has overcome death, so that he could make us share in the righteousness which he had obtained for us by his death. Second, by his power we too are raised up to a new life. Third, Christ's resurrection is to us a sure pledge of our glorious resurrection.%
}

%% ============================================================
%% LORD'S DAY XVIII
%% ============================================================

\vspace{0.5em}
\lordsday{18}

\catechismqa{46}{%
What do you confess when you say, he ascended into heaven?%
}{%
That Christ, before the eyes of his disciples, was taken up from the earth into heaven, and that he is there for our benefit until he comes again to judge the living and the dead.%
}

\catechismqa{47}{%
Is Christ, then, not with us until the end of the world, as he has promised us?%
}{%
Christ is true man and true God. With respect to his human nature he is no longer on earth, but with respect to his divinity, majesty, grace, and Spirit he is never absent from us.%
}

\catechismqa{48}{%
But are the two natures in Christ not separated from each other if his human nature is not present wherever his divinity is?%
}{%
Not at all, for his divinity has no limits and is present everywhere. So it must follow that his divinity is indeed beyond the human nature which he has taken on and nevertheless is within this human nature and remains personally united with it.%
}

\catechismqa{49}{%
How does Christ's ascension into heaven benefit us?%
}{%
First, he is our Advocate in heaven before his Father. Second, we have our flesh in heaven as a sure pledge that he, our Head, will also take us, his members, up to himself. Third, he sends us his Spirit as a counter-pledge, by whose power we seek the things that are above, where Christ is, seated at the right hand of God, and not the things that are on earth.%
}

%% ============================================================
%% LORD'S DAY XIX
%% ============================================================

\vspace{0.5em}
\lordsday{19}

\catechismqa{50}{%
Why is it added, and sits at the right hand of God?%
}{%
Christ ascended into heaven to manifest himself there as Head of his church, through whom the Father governs all things.%
}

\catechismqa{51}{%
How does the glory of Christ, our Head, benefit us?%
}{%
First, by his Holy Spirit he pours out heavenly gifts upon us, his members. Second, by his power he defends and preserves us against all enemies.%
}

\catechismqa{52}{%
What comfort is it to you that Christ will come to judge the living and the dead?%
}{%
In all my sorrow and persecution I lift up my head and eagerly await as judge from heaven the very same person who before has submitted himself to the judgment of God for my sake, and has removed all the curse from me. He will cast all his and my enemies into everlasting condemnation, but he will take me and all his chosen ones to himself into heavenly joy and glory.%
}
