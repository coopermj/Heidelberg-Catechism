%% grace-2.tex - Part 2b: Of Man's Redemption (God the Holy Spirit)
%% Questions 53-85, Lord's Days XX-XXXI

%% ============================================================
%% LORD'S DAY XX
%% ============================================================

\lordsday{20}

\catechismqa{53}{%
What do you believe concerning the Holy Spirit?%
}{%
First, he is, together with the Father and the Son, true and eternal God. Second, he is also given to me, to make me by true faith share in Christ and all his benefits, to comfort me, and to remain with me forever.%
}

%% ============================================================
%% LORD'S DAY XXI
%% ============================================================

\vspace{0.5em}
\lordsday{21}

\catechismqa{54}{%
What do you believe concerning the holy catholic church?%
}{%
I believe that the Son of God, out of the whole human race, from the beginning of the world to its end, gathers, defends, and preserves for himself, by his Spirit and Word, in the unity of the true faith, a church chosen to everlasting life. And I believe that I am and forever shall remain a living member of it.%
}

\catechismqa{55}{%
What do you understand by the communion of saints?%
}{%
First, that believers, all and everyone, as members of Christ have communion with him and share in all his treasures and gifts. Second, that everyone is bound to use his gifts readily and cheerfully for the benefit and well-being of the other members.%
}

\catechismqa{56}{%
What do you believe concerning the forgiveness of sins?%
}{%
I believe that God, because of Christ's satisfaction, will no longer remember my sins, nor my sinful nature, against which I have to struggle all my life, but will graciously grant me the righteousness of Christ, that I may never come into condemnation.%
}

%% ============================================================
%% LORD'S DAY XXII
%% ============================================================

\vspace{0.5em}
\lordsday{22}

\catechismqa{57}{%
What comfort does the resurrection of the body offer you?%
}{%
Not only shall my soul after this life immediately be taken up to Christ, my Head, but also this my flesh, raised by the power of Christ, shall be reunited with my soul and made like the glorious body of Christ.%
}

\catechismqa{58}{%
What comfort do you receive from the article about life everlasting?%
}{%
Even as I now already feel in my heart the beginning of eternal joy, so after this life I shall have perfect blessedness, such as no eye has seen, nor ear heard, nor the heart of man conceived---a blessedness in which to praise God forever.%
}

%% ============================================================
%% LORD'S DAY XXIII
%% ============================================================

\vspace{0.5em}
\lordsday{23}

\catechismqa{59}{%
But what does it help you now that you believe all this?%
}{%
In Christ I am righteous before God and an heir to life everlasting.%
}

\catechismqa{60}{%
How are you righteous before God?%
}{%
Only by true faith in Jesus Christ. Although my conscience accuses me that I have grievously sinned against all God's commandments, have never kept any of them, and am still inclined to all evil, yet God, without any merit of my own, out of mere grace, grants and credits to me the perfect satisfaction, righteousness, and holiness of Christ, as if I had never sinned nor been a sinner, and as if I had been as perfectly obedient as Christ was obedient for me---if only I accept this gift with a believing heart.%
}

\catechismqa{61}{%
Why do you say that you are righteous only by faith?%
}{%
Not that I am acceptable to God on account of the worthiness of my faith, for only the satisfaction, righteousness, and holiness of Christ is my righteousness before God. I can receive this righteousness and make it my own by faith only.%
}

%% ============================================================
%% LORD'S DAY XXIV
%% ============================================================

\vspace{0.5em}
\lordsday{24}

\catechismqa{62}{%
But why can our good works not be our righteousness before God, or at least a part of it?%
}{%
Because the righteousness which can stand before God's judgment must be absolutely perfect and wholly in agreement with the law of God, whereas even our best works in this life are all imperfect and defiled with sin.%
}

\catechismqa{63}{%
But do our good works earn nothing, even though God promises to reward them in this life and the next?%
}{%
This reward is not earned; it is a gift of grace.%
}

\catechismqa{64}{%
Does this teaching not make people careless and wicked?%
}{%
No. It is impossible that those who are grafted into Christ by true faith should not bring forth fruits of thankfulness.%
}

%% ============================================================
%% LORD'S DAY XXV
%% ============================================================

\vspace{0.5em}
\lordsday{25}

\catechismqa{65}{%
Since then faith alone makes us share in Christ and all his benefits, where does this faith come from?%
}{%
From the Holy Spirit, who works it in our hearts by the preaching of the gospel, and strengthens it by the use of the sacraments.%
}

\catechismqa{66}{%
What are the sacraments?%
}{%
The sacraments are visible, holy signs and seals instituted by God so that by their use he may the more fully declare and seal to us the promise of the gospel: that because of the one sacrifice of Christ accomplished on the cross he grants us freely the forgiveness of all our sins and eternal life.%
}

\catechismqa{67}{%
Are both the Word and the sacraments intended to focus our faith on the sacrifice of Jesus Christ on the cross as the only ground of our salvation?%
}{%
Yes, indeed. The Holy Spirit teaches us in the gospel and assures us by the sacraments that our entire salvation rests on the one sacrifice of Christ made for us on the cross.%
}

\catechismqa{68}{%
How many sacraments has Christ instituted in the new covenant?%
}{%
Two: holy baptism and the holy supper.%
}

%% ============================================================
%% LORD'S DAY XXVI
%% ============================================================

\vspace{0.5em}
\lordsday{26}

\catechismqa{69}{%
How does holy baptism signify and seal to you that the one sacrifice of Christ on the cross benefits you?%
}{%
In this way: Christ instituted this outward washing and with it gave the promise that, as surely as water washes away the dirt from the body, so certainly his blood and Spirit wash away the impurity of my soul, that is, all my sins.%
}

\catechismqa{70}{%
What does it mean to be washed with Christ's blood and Spirit?%
}{%
To be washed with Christ's blood means to receive forgiveness of sins from God, through grace, because of Christ's blood, poured out for us in his sacrifice on the cross. To be washed with his Spirit means to be renewed by the Holy Spirit and sanctified to be members of Christ, so that more and more we become dead to sin and lead a holy and blameless life.%
}

\catechismqa{71}{%
Where has Christ promised that we are washed with his blood and Spirit as surely as we are washed with the water of baptism?%
}{%
In the institution of baptism, where he says: Go therefore and make disciples of all nations, baptizing them in the name of the Father and of the Son and of the Holy Spirit. He who believes and is baptized will be saved, but he who does not believe will be condemned. This promise is repeated where Scripture calls baptism the washing of regeneration and the washing away of sins.%
}

%% ============================================================
%% LORD'S DAY XXVII
%% ============================================================

\vspace{0.5em}
\lordsday{27}

\catechismqa{72}{%
Does this outward washing with water itself wash away sins?%
}{%
No, only the blood of Jesus Christ and the Holy Spirit cleanse us from all sins.%
}

\catechismqa{73}{%
Why then does the Holy Spirit call baptism the washing of regeneration and the washing away of sins?%
}{%
God speaks in this way for a good reason: He wants to teach us that the blood and Spirit of Christ remove our sins just as water takes away dirt from the body. But, more important, he wants to assure us by this divine pledge and sign that we are as truly cleansed from our sins spiritually as we are bodily washed with water.%
}

\catechismqa{74}{%
Should infants also be baptized?%
}{%
Yes. Infants as well as adults belong to God's covenant and congregation. Through Christ's blood the redemption from sin and the Holy Spirit, who works faith, are promised to them no less than to adults. Therefore, by baptism, as sign of the covenant, they must be grafted into the Christian church and distinguished from the children of unbelievers. This was done in the old covenant by circumcision, in place of which baptism was instituted in the new covenant.%
}

%% ============================================================
%% LORD'S DAY XXVIII
%% ============================================================

\vspace{0.5em}
\lordsday{28}

\catechismqa{75}{%
How does the Lord's Supper signify and seal to you that you share in Christ's one sacrifice on the cross and all his benefits?%
}{%
In this way: Christ has commanded me and all believers to eat of this broken bread and drink of this cup in remembrance of him, and has joined with this command these promises: First, as surely as I see with my eyes the bread of the Lord broken for me and the cup given to me, so surely was his body offered for me and his blood poured out for me on the cross. Second, as surely as I receive from the hand of the minister and taste with my mouth the bread and the cup of the Lord as sure signs of Christ's body and blood, so surely does he himself nourish and refresh my soul to everlasting life with his crucified body and poured-out blood.%
}

\catechismqa{76}{%
What does it mean to eat the crucified body of Christ and to drink his poured-out blood?%
}{%
It means not only to embrace with a believing heart all the sufferings and the death of Christ, and thereby to receive forgiveness of sins and eternal life, but also to be so united more and more to his sacred body by the Holy Spirit, who dwells both in Christ and in us, that, although Christ is in heaven and we are on earth, we are nevertheless flesh of his flesh and bone of his bones, and we live and are governed forever by one Spirit, as the members of our body are by one soul.%
}

\catechismqa{77}{%
Where has Christ promised that he will nourish and refresh believers with his body and blood as surely as they eat of this broken bread and drink of this cup?%
}{%
In the institution of the Lord's Supper: The Lord Jesus on the night when he was betrayed took bread, and when he had given thanks, he broke it, and said, This is my body which is for you. Do this in remembrance of me. In the same way also the cup, after supper, saying, This cup is the new covenant in my blood. Do this, as often as you drink it, in remembrance of me. For as often as you eat this bread and drink the cup, you proclaim the Lord's death until he comes. This promise is repeated by Paul where he says: The cup of blessing which we bless, is it not a participation in the blood of Christ? The bread which we break, is it not a participation in the body of Christ? Because there is one bread, we who are many are one body, for we all partake of the one bread.%
}

%% ============================================================
%% LORD'S DAY XXIX
%% ============================================================

\vspace{0.5em}
\lordsday{29}

\catechismqa{78}{%
Do the bread and wine become the real body and blood of Christ?%
}{%
No. Just as the water of baptism is not changed into the blood of Christ and is not the washing away of sins itself but is simply God's sign and assurance, so also the bread in the Lord's Supper does not become the body of Christ itself, although it is called Christ's body in keeping with the nature and usage of sacraments.%
}

\catechismqa{79}{%
Why then does Christ call the bread his body and the cup his blood, or the new covenant in his blood, and why does Paul speak of a participation in the body and blood of Christ?%
}{%
Christ speaks in this way for a good reason: He wants to teach us that just as bread and wine sustain us in this temporal life, so his crucified body and poured-out blood are the true food and drink of our souls for eternal life. But, more important, by this visible sign and pledge he wants to assure us that through the working of the Holy Spirit we share in his true body and blood as surely as our mouths receive these holy signs in his remembrance, and that all his sufferings and his death are our own as certainly as if we ourselves had suffered and paid for our sins.%
}

%% ============================================================
%% LORD'S DAY XXX
%% ============================================================

\vspace{0.5em}
\lordsday{30}

\catechismqa{80}{%
What difference is there between the Lord's Supper and the papal mass?%
}{%
The Lord's Supper testifies to us, first, that we have complete forgiveness of all our sins through the one sacrifice of Jesus Christ, which he himself accomplished on the cross once for all; and, second, that through the Holy Spirit we are grafted into Christ, who with his true body is now in heaven at the right hand of the Father, and this is where he wants to be worshipped. But the mass teaches, first, that the living and the dead do not have forgiveness of sins through the sufferings of Christ unless Christ is still offered for them daily by the priests; and, second, that Christ is bodily present in the form of bread and wine, and there is to be worshipped. Therefore the mass is basically nothing but a denial of the one sacrifice and suffering of Jesus Christ, and an accursed idolatry.%
}

\catechismqa{81}{%
Who should come to the table of the Lord?%
}{%
Those who are truly displeased with themselves because of their sins and yet trust that these are forgiven them and that their remaining weakness is covered by the suffering and death of Christ, and who also desire more and more to strengthen their faith and amend their life. But hypocrites and those who do not repent eat and drink judgment upon themselves.%
}

\catechismqa{82}{%
Should those be admitted to the Lord's Supper who by their confession and life show that they are unbelieving and ungodly?%
}{%
No, for then the covenant of God would be profaned and his wrath kindled against the whole congregation. Therefore, according to the command of Christ and his apostles, the Christian church is duty-bound to exclude such persons by the keys of the kingdom of heaven, until they amend their lives.%
}

%% ============================================================
%% LORD'S DAY XXXI
%% ============================================================

\vspace{0.5em}
\lordsday{31}

\catechismqa{83}{%
What are the keys of the kingdom of heaven?%
}{%
The preaching of the holy gospel and church discipline. By these two the kingdom of heaven is opened to believers and closed to unbelievers.%
}

\catechismqa{84}{%
How is the kingdom of heaven opened and closed by the preaching of the gospel?%
}{%
According to the command of Christ, the kingdom of heaven is opened when it is proclaimed and publicly testified to each and every believer that God has really forgiven all their sins for the sake of Christ's merits, as often as they by true faith accept the promise of the gospel. The kingdom of heaven is closed when it is proclaimed and testified to all unbelievers and hypocrites that the wrath of God and eternal condemnation rest upon them as long as they do not repent. According to this testimony of the gospel, God will judge both in this life and in the life to come.%
}

\catechismqa{85}{%
How is the kingdom of heaven closed and opened by church discipline?%
}{%
According to the command of Christ, those who call themselves Christians but show themselves to be unchristian in doctrine or life are first repeatedly admonished in a brotherly manner. If they do not give up their errors or wickedness, they are reported to the church, that is, to the elders. If they do not heed also their admonitions, they are forbidden the use of the sacraments, and they are excluded by the elders from the Christian congregation, and by God himself from the kingdom of Christ. They are again received as members of Christ and of his church when they promise and show real amendment.%
}
