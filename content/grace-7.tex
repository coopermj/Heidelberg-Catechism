%% grace-7.tex - Part 2g: Of Man's Redemption (Lord's Supper and the Keys)
%% Questions 78-85, Lord's Days XXIX-XXXI

%% ============================================================
%% LORD'S DAY XXIX
%% ============================================================

\lordsday{29}

\catechismqa{78}{%
Do the bread and wine become the real body and blood of Christ?%
}{%
No. Just as the water of baptism is not changed into the blood of Christ and is not the washing away of sins itself but is simply God's sign and assurance, so also the bread in the Lord's Supper does not become the body of Christ itself, although it is called Christ's body in keeping with the nature and usage of sacraments.%
}

\catechismqa{79}{%
Why then does Christ call the bread his body and the cup his blood, or the new covenant in his blood, and why does Paul speak of a participation in the body and blood of Christ?%
}{%
Christ speaks in this way for a good reason: He wants to teach us that just as bread and wine sustain us in this temporal life, so his crucified body and poured-out blood are the true food and drink of our souls for eternal life. But, more important, by this visible sign and pledge he wants to assure us that through the working of the Holy Spirit we share in his true body and blood as surely as our mouths receive these holy signs in his remembrance, and that all his sufferings and his death are our own as certainly as if we ourselves had suffered and paid for our sins.%
}

%% ============================================================
%% LORD'S DAY XXX
%% ============================================================

\vspace{0.5em}
\lordsday{30}

\catechismqa{80}{%
What difference is there between the Lord's Supper and the papal mass?%
}{%
The Lord's Supper testifies to us, first, that we have complete forgiveness of all our sins through the one sacrifice of Jesus Christ, which he himself accomplished on the cross once for all; and, second, that through the Holy Spirit we are grafted into Christ, who with his true body is now in heaven at the right hand of the Father, and this is where he wants to be worshipped. But the mass teaches, first, that the living and the dead do not have forgiveness of sins through the sufferings of Christ unless Christ is still offered for them daily by the priests; and, second, that Christ is bodily present in the form of bread and wine, and there is to be worshipped. Therefore the mass is basically nothing but a denial of the one sacrifice and suffering of Jesus Christ, and an accursed idolatry.%
}

\catechismqa{81}{%
Who should come to the table of the Lord?%
}{%
Those who are truly displeased with themselves because of their sins and yet trust that these are forgiven them and that their remaining weakness is covered by the suffering and death of Christ, and who also desire more and more to strengthen their faith and amend their life. But hypocrites and those who do not repent eat and drink judgment upon themselves.%
}

\catechismqa{82}{%
Should those be admitted to the Lord's Supper who by their confession and life show that they are unbelieving and ungodly?%
}{%
No, for then the covenant of God would be profaned and his wrath kindled against the whole congregation. Therefore, according to the command of Christ and his apostles, the Christian church is duty-bound to exclude such persons by the keys of the kingdom of heaven, until they amend their lives.%
}

%% ============================================================
%% LORD'S DAY XXXI
%% ============================================================

\vspace{0.5em}
\lordsday{31}

\catechismqa{83}{%
What are the keys of the kingdom of heaven?%
}{%
The preaching of the holy gospel and church discipline. By these two the kingdom of heaven is opened to believers and closed to unbelievers.%
}

\catechismqa{84}{%
How is the kingdom of heaven opened and closed by the preaching of the gospel?%
}{%
According to the command of Christ, the kingdom of heaven is opened when it is proclaimed and publicly testified to each and every believer that God has really forgiven all their sins for the sake of Christ's merits, as often as they by true faith accept the promise of the gospel. The kingdom of heaven is closed when it is proclaimed and testified to all unbelievers and hypocrites that the wrath of God and eternal condemnation rest upon them as long as they do not repent. According to this testimony of the gospel, God will judge both in this life and in the life to come.%
}

\catechismqa{85}{%
How is the kingdom of heaven closed and opened by church discipline?%
}{%
According to the command of Christ, those who call themselves Christians but show themselves to be unchristian in doctrine or life are first repeatedly admonished in a brotherly manner. If they do not give up their errors or wickedness, they are reported to the church, that is, to the elders. If they do not heed also their admonitions, they are forbidden the use of the sacraments, and they are excluded by the elders from the Christian congregation, and by God himself from the kingdom of Christ. They are again received as members of Christ and of his church when they promise and show real amendment.%
}
